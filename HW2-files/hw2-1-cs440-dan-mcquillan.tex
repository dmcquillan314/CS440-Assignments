\documentclass[12pt,letterpaper]{article}

\newenvironment{proof}{\noindent{\bf Proof:}}{\qed\bigskip}

\newtheorem{theorem}{Theorem}
\newtheorem{corollary}{Corollary}
\newtheorem{lemma}{Lemma} 
\newtheorem{claim}{Claim}
\newtheorem{fact}{Fact}
\newtheorem{definition}{Definition}
\newtheorem{assumption}{Assumption}
\newtheorem{observation}{Observation}
\newtheorem{example}{Example}
\newcommand{\qed}{\rule{7pt}{7pt}}

\newcommand{\assignment}[4]{
\thispagestyle{plain} 
\newpage
\setcounter{page}{1}
\noindent
\begin{center}
\framebox{ \vbox{ \hbox to 6.28in
{\bf CS440: Introduction to Artificial Intelligence \hfill #1}
\vspace{4mm}
\hbox to 6.28in
{\hspace{2.5in}\large\mbox{In class assignment #2}}
\vspace{4mm}
\hbox to 6.28in
{{\it Handed Out: #3 \hfill Due: #4}}
}}
\end{center}
}

\newcommand{\solution}[4]{
\thispagestyle{plain} 
\newpage
\setcounter{page}{1}
\noindent
\begin{center}
\framebox{ \vbox{ \hbox to 6.28in
{\bf CS440 : Introduction to Artificial Intelligence \hfill #4}
\vspace{4mm}
\hbox to 6.28in
{\hspace{2.5in}\large\mbox{In class assignment #3}}
\vspace{4mm}
\hbox to 6.28in
{#1 \hfill {\it Handed In: #2}}
}}
\end{center}
\markright{#1}
}

\newenvironment{algorithm}
{\begin{center}
\begin{tabular}{|l|}
\hline
\begin{minipage}{1in}
\begin{tabbing}
\quad\=\qquad\=\qquad\=\qquad\=\qquad\=\qquad\=\qquad\=\kill}
{\end{tabbing}
\end{minipage} \\
\hline
\end{tabular}
\end{center}}

\def\Comment#1{\textsf{\textsl{$\langle\!\langle$#1\/$\rangle\!\rangle$}}}


\usepackage{algorithm}
\usepackage{listings}
%\usepackage{algpseudocode}
\usepackage{graphicx,amssymb,amsmath}
\usepackage{epstopdf}
\sloppy


\oddsidemargin 0in
\evensidemargin 0in
\textwidth 6.5in
\topmargin -0.5in
\textheight 9.0in

\begin{document}

\solution{Dan McQuillan}{\today}{2 - Part 1}{Spring 2014}

\pagestyle{myheadings}  % Leave this command alone

\begin{enumerate}

	\item {\bf Solution to problem 1}
	
		\begin{enumerate}
			\item{\bf Training on data set 1}
				\begin{enumerate}
					\item[a)] The final weights \\ \\
						\(Threshhold = 0.0\) \\ \\
						\bf Weights: \\
						\(w_0 \rightarrow -748.0\) \\
						\(w_1 \rightarrow 4775.0\) \\
						\(w_2 \rightarrow 17744.0\) \\
						\(w_3 \rightarrow 780.0\) \\
						\(w_4 \rightarrow -6.0\) \\
						\(w_5 \rightarrow 657.0\) \\
						\(w_6 \rightarrow 3041.0\) \\
						\(w_7 \rightarrow -2008.0\) \\
						\(w_8 \rightarrow 5501.0\) \\
						\(w_9 \rightarrow 9662.399999998896\) \\
						\(w_{10} \rightarrow 3846.0\) \\
						\(w_{11} \rightarrow 14912.0\) \\
						\(w_{12} \rightarrow 21058.0\) \\
						
					\item[b)] The number of training epochs required \\
						\textnormal{The training took 1975 epochs.} \\
					
					\item[c)] The margin \\
						\(\gamma \rightarrow 2.206549570044712 * 10^{-5}\)
				\end{enumerate}
			\item{\bf Test on data set 1}
				\begin{enumerate}
					\item[a)] A confusion matrix \\
						\( 
							\begin{array}{|c|c|}
								\hline
								54 & 0 \\
								\hline
								0 & 63 \\
								\hline
							\end{array} 
						\) \\
					\item[b)] Two lists of example indices \\ 
						No errors were found with this dataset since the weight vectors were calculated from this dataset. \\
					\item[c)] The total loss summed over the misclassified examples \\
						The loss for this dataset is 0.0 since it is the training data set.					\\
				\end{enumerate}
			\item{\bf Test on data set 2}
				\begin{enumerate}
					\item[a)] A confusion matrix \\
						\( 
							\begin{array}{|c|c|}
								\hline
								12 & 1 \\
								\hline
								0 & 20 \\
								\hline
							\end{array} 
						\) \\
					\item[b)] Two lists of example indices \\
						\bf{False Negatives:} \\
						\textnormal{Index: 22} \\
						\textnormal{Inputs:} \( (60.0, 1.0, 4.0, 140.0, 293.0, 0.0, 2.0, 170.0, 0.0, 1.2, 2.0, 2.0, 7.0) \) \\ \\
						\bf{False Positives:} \\
						\textnormal{There were no false positives} \\
					\item[c)] The total loss summed over the misclassified examples \\
					\textnormal{Total loss:} \( 448.1200000013341\)
				\end{enumerate}
			\item{\bf Application to data set 3} \\
				\bf{Classifications: } \\
				\( 0 \rightarrow 0.0 \) \\
				\( 1 \rightarrow 1.0 \) \\
				\( 2 \rightarrow 0.0 \) \\
				\( 3 \rightarrow 1.0 \) \\
				\( 4 \rightarrow 1.0 \) \\
				\( 5 \rightarrow 0.0 \) \\
				\( 6 \rightarrow 0.0 \) \\
				\( 7 \rightarrow 0.0 \) \\
				\( 8 \rightarrow 0.0 \) \\
				\( 9 \rightarrow 1.0 \) \\
				\( 10 \rightarrow 1.0 \) \\
				\( 11 \rightarrow 0.0 \) \\
				\( 12 \rightarrow 1.0 \) \\
				\( 13 \rightarrow 0.0 \) \\
				\( 14 \rightarrow 0.0 \) \\
				\( 15 \rightarrow 0.0 \) \\
				\( 16 \rightarrow 1.0 \) \\
				\( 17 \rightarrow 0.0 \) \\
				\( 18 \rightarrow 0.0 \) \\
				\( 19 \rightarrow 0.0 \) \\
				\( 20 \rightarrow 0.0 \) \\ \\
				
				\bf{Which feature is the most important?} \\
				\textnormal{In order to determine the most influential property I have used the correlation coefficients to find the most correlation between each property and the classification.  The source code to find this property is also within the project.} \\ \\
				\( \begin{array}{l|l}
				\bf{Label} & \bf{Correlation Coefficient} \\
				\hline
				\text{ age } & 0.576399638726158 \\
				\text{ sex } & 0.15811388300841894 \\
				\text{ chest } & 0.4934637712198269 \\
				\text{ resting blood pressure } & 0.08008953852726183 \\
				\text{ serum cholestoral } & -0.32382553481251514 \\
				\text{ fasting blood sugar } & 0.31622776601683766 \\
				\text{ resting electrocardiographic results } & 0.1386750490563073 \\
				\text{ maximum heart rate achieved } & -0.5353426981014223 \\
				\text{ exercise induced angina } & 0.685994340570035 \\
				\text{ oldpeak } & 0.8086701966434094 \\
				\text{ slope } & 0.6123724356957945 \\
				\text{ number of major vessels } & 0.8152133857595864 \\
				\text{ thal } & 0.43905703995876144 \\
				\end{array} \) \\ \\
				
				\bf{ Property With Maximum Correlation} \\ 
				Label:  \textnormal{number of major vessels} \\
				Correlation Coefficient: \(0.8152133857595864\)
		\end{enumerate}

\end{enumerate}



\end{document}

